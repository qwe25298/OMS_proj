\include{preamble}

\newtotcounter{citenum} %From the package documentation
\def\oldbibitem{}
\let\oldbibitem=\bibitem
\def\bibitem{\stepcounter{citenum}\oldbibitem}

\begin{document}
\includepdf[pages={1}]{title.pdf}

\tableofcontents
\newpage
\section{Введение}
DNS--сервер (Domain Name System) является одной из ключевых составляющих сети Интернет. Он выполняет функцию перевода доменных имен в IP--адреса,
что позволяет пользователям использовать удобочитаемые адреса вместо числовых идентификаторов для доступа к ресурсам в сети. DNS--серверы играют
важную роль в обеспечении стабильной и безопасной работы интернет-инфраструктуры, поэтому их изучение и оптимизация имеют большое значение для
специалистов в области информационных технологий.

Для исследования процесса его работы удобно использовать \textbf{очередь}. В данном случае для задачи имитации сервера об очереди будет рассуждать как
о структуре с доступом к элементам по принципу "первый вошел --- первый вышел" (поскольку при поступлении на сервер запросы обрабатываются в том порядке
, в котором они поступали). Тем не менее, любое цифровое устройство ограничено запасом памяти.

В данной курсовой работе моделирование работы сервера и его ограничение по памяти будет производиться за счет двух очередей: одна будет
представлять сам сервер, обрабатывающий запросы, а вторая --- жесткий диск, на который запросы отправляются в том случае, если они поступили на сервер в момент,
когда очередь обработки на нем переполнена.

Целью данной курсовой работы является определение оптимальной пропускной способности для очереди на сервере в зависимости от интенсивности запросов в
один из трех условных периодов дня: ночь (00:00 -- 9:00), день (9:00 -- 17:00) и вечер (17:00 -- 00:00).

В ходе работы необходимо решить следующие задачи:
\begin{enumerate}
    \item Поставить задачу, которая позволила бы построить модель для исследования пропускной способности сервера
    \item Построить имитационную модель и провести эксперименты с среде моделирования AnyLogic
    \item Сделать выводы, основываясь на проведенных экспериментах
\end{enumerate}

\newpage
\section{Постановка задачи и логическая модель}
\subsection{Условие}
Для проведения эксперимента поставим задачу следующим образом.

На DNS сервер поступают запросы. Длины запросов имеют равномерное распределение с нижней границей
в 10 и верхней в 500 байт. Одновременно в канал сервера может поступать А запросов в секунду. Обработка одного
запроса занимает С миллисекунд. Для необработанных запросов создается очередь, которая записывается
на жесткий диск сервера емкостью B байт. Интенсивность запросов меняется в зависимости от времени суток:
\begin{itemize}
    \item День: в среднем D зап./с
    \item Вечер: V зап./с
    \item Ночь: N зап./с
\end{itemize}

При этом N < D < V. Компьютер клиента может ожидать ответ на запрос в течение 30 секунд.

Задача: определить оптимальную пропускную способность А . Пропускная способность считается оптимальной, когда она минимальна, при условии,
что диск занят не более чем на 70 процентов. Максимальная вместимость диска 500 * 1000 / C * 30  байт.

\subsection{Блок--схема представления модели}
В концептуальном виде модели системы DNS сервера представлены на рис \ref{concept}.

\begin{figure} [h]
    \center{\includegraphics[width=0.9\linewidth]{concept.png}}
    \caption{Логическая блок--схема разработанной модели}
    \label{concept}
\end{figure}

Составляющие приведенной схемы представляют следующее:
\begin{itemize}
    \item Параллелограмм --- точка входа в модель системы. Представляет собой централизованное
    место прибытия новых запросов на сервер из различных источников.
    \item Ромб --- развилка, соответствует выбору или определению способа дальнейшей
    обработки запроса в зависимости от текущего состояния системы
    \item Прямоугольник --- определяет процессы, происходящие с поступившим запросом.
    В рамках составленной модели - обработка запроса или ожидание в очереди. 
    \item Круг --- конечные этапы жизненного цикла запроса в рамках созданной модели.
    Обозначают непосредственные результаты обработки запроса, успешные или нет.
\end{itemize}
\newpage

\section{Создание модели в среде AnyLogic}
\begin{figure} [h]
    \center{\includegraphics[width=0.9\linewidth]{model.png}}
    \caption{реализация разработанной модели DNS--сервера в среде AnyLogic}
\end{figure}


\subsection{Описание семантики элементов}
При построении дискретно--событийной модели применялись следующие
элементы палитры блока «Библиотека моделирования процессов» из стандартного
пакета AnyLogic:
\begin{itemize}
    \item Source –-- тип блока, являющийся точкой входа модели. Предназначен для
    создания потока агентов. В данной модели представлен в виде единственного
    блока «start» для порождения агентов типа "Запрос".
    \item Queue –-- тип блока, реализующий очередь. Предназначен для управления
    потоком агента, формирования блока с входом и выходом типа FIFO. В
    данной модели используется дли имитирования очереди запросов на
    обработку в сервере ("Channel"), а также для имитации памяти и очереди на жестком
    диске (hardware).
    \item Delay –-- тип блока, реализующий временную задержку. Предназначен для
    приостановки хода движения агента. В данной модели представлен в виде
    блока "Process", которые содержат в себе семантику обработки запросов
    сервером.
    \item SelectOutput --- тип блока, реализующий логическое ветвление.
    Предназначен для ветвления хода движения агентов по вероятности или
    условию. Представлен в модели блоками "isFreeQueue" и "isFreeMemory".
    Первый реализует развилку по наличию свободного места в очереди запросов на сервер,
    а второй проверяет наличие свободной памяти на жестком диске.
    \item Sink –-- тип блока, являющийся точкой выхода модели. По нему можно
    отслеживать количество агентов, дошедших до конца. В данной модели
    представлен тремя блоками: "Success", "Service\_Unavaliable\_503" и
    "A\_Timeout\_Occured\_524", уничтожающими агентов. Первый отвечает за уход запроса из системы при его
    успешной обработке, второй за ситуацию, когда запросы переполняют по памяти жесткий диск и
    сервер ложится, и последний за случай, когда запрос пробыл на жестком диске больше времени, чем это
    допустимо условиями задачи.
\end{itemize}

\subsection{Описание логики элементов}
В представленной модели, внутри всех блоков реализована дополнительная логика поведения
с использованием языка программирования Java, в соответствии с предоставляемыми AnyLogic
возможностями.

Условия внедрены в соответствии с концептуальной моделью системы. Данные внедрения
позволяют динамически вычислять такие, необходимые для правильной работы параметры,
как:
\begin{itemize}
    \item Динамически изменять оставшееся место на диске в зависимости от количества
    пришедших и записанных на него запросов, а также реализовывать логику
    ошибок сервера 503 и 524, при невозможности или несвоевременности
    обработки запроса.
    \begin{figure} [h]
        \center{\includegraphics[width=0.9\linewidth]{entries.png}}
        \caption{Настройки связки блоков IsFreeMemory и hardware ---
        обрабатывающие выделение/высвобождение места на диске.}
    \end{figure}
    \item Ветвить условия для распределения запросов по циклам их обработки, в зависимости
    от текущего состояния и конфигурации системы.
    \begin{figure} [h]
        \center{\includegraphics[width=0.9\linewidth]{branch.png}}
        \caption{Блок IsQueueFree --- отвечающий за отправку запроса в очередь
        на обработку либо же отправки на жесткий диск для ожидания.}
    \end{figure}

    \item Регулировать настройки системы во время ее работы, что является важным
    аспектом для проведения качественного исследования и определения наиболее оптимальных
    параметров, обозначенных в формулировке задачи. В данном случае регулируемыми
    параметрами являются:
    \begin{itemize}
        \item Скорость обработки запроса
        \item Максимально допустимый объем диска. Линейно зависит от скорости обработки
        запроса исходя из представленной в условии задачи формулы.
    \end{itemize}

    \begin{figure} [h]
        \center{\includegraphics[width=0.9\linewidth]{mem.png}}
        \caption{Настройки связки блоков IsQueueFree и Process,
        представленные на рисунке настройки позволяют реализовать
        изменение параметров системы в динамике}
    \end{figure}

    \item Настройки ухода по тайм--ауту в блоке hardware а также условие в блоке
    IsFreeMemory реализуют логику ошибок обработки запроса на сервере. В случае
    ухода по тайм--ауту ошибка 524 A Timeout Occurred --- превышен лимит ожидания,
    в случае несрабатывания условия в блоке IsFreeMemory --- 503 Service Unavailable,
    что говорит о переполненном диске и невозможности серверу обработать запрос.

    \begin{figure} [h]
        \center{\includegraphics[width=0.9\linewidth]{perm.png}}
        \caption{Настройки блоков hardware и IsFreeMemory соответственно}
    \end{figure}

    \item Динамически генерировать количество и размеры приходящих запросов. А
    именно длину, для определения занимаемого объема памяти на диске и интенсивность
    поступления запросов, что обеспечивает нелинейное поведение нагрузки на систему.
    \begin{figure} [h]
        \center{\includegraphics[width=0.9\linewidth]{source.png}}
        \caption{Настройки блока source, отвечающие за размер запроса, а именно,
        переменной size агента Запрос, а также случайную интенсивность поступления
        нагрузки соответственно.}
    \end{figure}
\end{itemize}

\newpage
\subsection{Вспомогательные элементы}
Для задания некоторых параметров модели и проведения экспериментов в модели использовались следующие вспомогательные элементы:
\begin{enumerate}
    \item Переменная \textit{maxMemory} --- переменная, необходимая для возможности динамически задавать максимальный размер жесткого диска (
    максимальный размер ЖД был сделан динамическим в целях упрощения проведения серий экспериментов, чтобы не приходилось задавать его вручную
    в зависимости от интенсивности, в момент проведения каждого эксперимента maxMemory является фиксированной величиной)
    \item Переменная \textit{memory} --- переменная, отражающая количество оставшегося места на диске с учетом пришедшего в него на очередь запроса.
    \item Переменная \textit{sizeOfChannel} --- переменная, описывающая искомую величину --- пропускную способность сервера.
    \item Переменная \textit{usedSpace} --- переменная, которая описывает занятое место на диске с учетом пришедшего запроса.
    \item Агент \textit{Запрос} --- транзакт, играющий роль запроса и содержащий информацию о размере этого запроса в байтах.
\end{enumerate}

\subsection{Элементы визуализации}

\newpage
\section{Имитационный эксперимент}
По построенной модели проводился имитационный эксперимент с целью определения наиболее
оптимальной пропускной способности сервера в зависимости от интенсивности.
В качестве демонстрационных можно выделить следующие эксперименты:
\begin{itemize}
    \item djgoigjo
    \item gjojgoitjg
    \item gjojgoitj
    \item rniotuwjgoi
    \item fbrtugg
\end{itemize}



\newpage
\section{Заключение}



\end{document}